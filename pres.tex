\documentclass{beamer}

\usepackage[utf8]{inputenc}
\usepackage[frenchb]{babel}
\usepackage[T1]{fontenc}

\usetheme{Singapore}

\begin{document}
\title{Analyse Lexicale et Syntaxique\\
	   } 
\author{Ken Déguernel - François Deslandes} 
\date{\today} 

\frame{\titlepage} 

\frame{\frametitle{Table of contents}\tableofcontents} 


\section{Présentation} 
\begin{frame}
\frametitle{Objectifs}
L'objectif de ce projet est d'appliquer les outils étudiés lors du cours d'Analyse Lexicale et Syntaxique à un exemple concret.\\
Le projet se décompose en deux parties :
\begin{enumerate}
\item Analyse Lexicale : comportant une partie de nettoyage et une partie de reconnaissance d'un texte. \\
\item Analyse Syntaxique : consistant en une vérification de l'intégrité et de la cohérence du texte. \\
\end{enumerate}
\end{frame}

\begin{frame}
\frametitle{Sujet}
Nous souhaitons produire de manière automatique des fichiers de partitions de jazz en format pdf, à partir d'un fichier source en texte brut.\\
Le fichier source comprend un certain nombre d'informations sur le morceau :
\begin{itemize}
\item Nom du morceau, auteur, style,
\item Mesuration, tonalité,
\item Thèmes (liste d'accords).
\end{itemize}
\end{frame}

\begin{frame}[fragile]
\frametitle{Exemple}
\begin{columns}

\column{1.5in}
\begin{tiny}
\begin{verbatim}
TITLE "Space Oddity" .Accompagnement guitare 
COMPOSER "David Bowie"
STYLE "Rock"
MESURATION 4/4
TONE "C"
PROG "AAB"

A/
C|Emin|C|Emin|
Amin|C|D7|D7|
\A


B/
C|E7|F|Fmin|
C|F|Fmin|F|
Fmaj7|Emin7|Fmaj7|Emin7|
Bb|Amin|G|F|
\B
\end{verbatim}
\end{tiny}

\column{1.5in}
\begin{center}
\textbf{Space Oddity}
\end{center}
\begin{small}
\begin{center}
\textbf{David Bowie}
\end{center}


Style : Rock

Mesuration : 4/4

Tone : C

Progression : AAB
\end{small}

\textbf{A}
\begin{small}
\begin{tabular}{|c|c|c|c|}
\hline
C & Emin & C & Emin \\
\hline
Amin & C & D7 & D7 \\
\hline
\end{tabular}
\end{small}

\textbf{B}
\begin{small}
\begin{tabular}{|c|c|c|c|}
\hline
C & E7 & F & Fmin \\
\hline
C & F & Fmin & F \\
\hline
Fmaj7 & Emin7 & Fmaj7 & Emin7 \\
\hline
Bb & Amin & G & F \\
\hline
\end{tabular}
\end{small}

\end{columns}
\end{frame}



\section{Analyse lexicale}
\begin{frame}
\frametitle{Analyse lexicale}
L'analyse lexicale consiste à :
\begin{enumerate}
\item Nettoyer le texte (commentaires, espacements surnuméraires, lignes vides,...),\\
\item Repérer les informations du morceau (Titre, Compositeur,...),\\
\item Reconnaître les thèmes et les accords,\\
\item Pouvoir reconnaître du texte,
\item Pouvoir retourner les valeurs lues.
\end{enumerate}
\end{frame}

\begin{frame}[fragile]
\frametitle{Reconnaître les informations}

On repère les identificateur d'expressions et on retourne le token correspondant à l'analyseur syntaxique.
\begin{tiny}
\begin{verbatim}
TITLE         {return TITLE;}
COMPOSER      {return COMPOSER;}
MESURATION    {return MESURATION;}
TONE          {return TONE;}
PROG          {return PROG;}
STYLE         {return STYLE;}
\end{verbatim}
\end{tiny}

On repère le texte entre guillemets et on le retourne dans la variables \textit{yylval}.
\begin{tiny}
\begin{verbatim}
TEXT        \"([a-z]|[ ])*\"
{TEXT}        {yylval=strndup(yytext+1, strlen(yytext)-2);return TEXT;}
\end{verbatim}
\end{tiny}


\end{frame}

\begin{frame}[fragile]
\frametitle{Reconnaître les grilles d'accords}
Un accord est composé d'une basse (une note de A à G) et d'une nature (par exemple : mineure, majeure,...)
\footnote{\tiny{La notation A-G est la notation américaine pour les notes do,ré,mi,...}\\}
\begin{tiny}
\begin{verbatim}
NATURE      (Maj|min|sus4|alt|aug|6|7|9|11|13|b|#|o|-|\+)
CHORD       ([A-G]{NATURE}*|%)
MES         [0-9][/][0-9]
PROG        \"[A-B]*\"
TONEVAL     \"[A-G]\"
\end{verbatim}
\end{tiny}

Quand on reconnaît un accord, on le retourne à l'analyseur syntaxique dans \textit{yylval}.
\begin{tiny}
\begin{verbatim}
{MES}             {yylval=strdup(yytext);return MES;}
{PROG}            {yylval=strndup(yytext+1, strlen(yytext)-2);return PROGVAL;}
{TONEVAL}         {yylval=strndup(yytext+1, strlen(yytext)-2);return TONEVAL;}
{CHORD}           {yylval=strdup(yytext);return CHORD;}
\end{verbatim}
\end{tiny}

\end{frame}




\section{Analyse syntaxique}
\begin{frame}[fragile]
\frametitle{Analyse syntaxique}
L'analyse syntaxique du texte source consiste à repérer les différents éléments dans un ordre donné et avec une forme donnée.

\begin{tiny}
\begin{verbatim}
header:
	|title composer style mesuration tone prog  themes
	;

themes:
	|theme_a theme_b
	;

chords:
	|chords chord_line
	;
\end{verbatim}
\end{tiny}

\end{frame}



\begin{frame}[fragile]
\frametitle{Analyse syntaxique}
On repère les informations grâce à leurs \textit{tokens} et on traduit en Latex. On utilise les valeurs retournées par l'analyseur lexical dans \textit{yylval}.

\begin{columns}
\column{1.5in}
\begin{tiny}
\begin{verbatim}
title:
	TITLE TEXT
	{
		printf("\\begin{center}\n"
				"\\textbf{{\\huge %s}}\n"
				"\\end{center}\n\n",
				$2);
	}
	;
\end{verbatim}
\end{tiny}

\column{1.5in}
\begin{tiny}
\begin{verbatim}
theme_b:
	theme_b_start chords theme_b_stop
	{
		printf("\n\n");
	}
	;
theme_b_start:
	THEME_B_START
	{
		printf("\\begin{huge}\n"
					"\\section*{B}\n"
					"\\begin{tabular}{|c|c|c|c|}\n"
		    		"\\hline\n");
	}
	;
chord_line:
	CHORD SEP CHORD SEP CHORD SEP CHORD SEP
	{
		printf("%s & %s & %s & %s \\\\\n"
					"\\hline\n", $1, $3, $5, $7);
	}
	;
\end{verbatim}
\end{tiny}

\end{columns}
\end{frame}

\section{Mise en forme}
\begin{frame}
\frametitle{Mise en forme}
L'intégration de la mise en forme Latex est assez simple.
\begin{enumerate}
\item Avant l'analyse du texte : Ecrire le header Latex {\small (begin document, packages, etc)}.
\item Pendant l'analyse du texte : Ecrire les informations dans leur format Latex {\small (sections, titres, polices, etc)}.
\item Après l'analyse du texte : Ecrire le footer Latex {\small (end document)}. 
\end{enumerate}
\end{frame}

\section{Résultats}
\begin{frame}[fragile]
\frametitle{Résultats}

\begin{columns}

\column{1.5in}
\begin{tiny}
\begin{verbatim}
TITLE "Space Oddity" .Accompagnement guitare 
COMPOSER "David Bowie"
STYLE "Rock"
MESURATION 4/4
TONE "C"
PROG "AAB"

A/
C|Emin|C|Emin|
Amin|C|D7|D7|
\A


B/
C|E7|F|Fmin|
C|F|Fmin|F|
Fmaj7|Emin7|Fmaj7|Emin7|
Bb|Amin|G|F|
\B
\end{verbatim}
\end{tiny}

\column{1.5in}
\begin{center}
\textbf{Space Oddity}
\end{center}
\begin{small}
\begin{center}
\textbf{David Bowie}
\end{center}


Style : Rock

Mesuration : 4/4

Tone : C

Progression : AAB
\end{small}

\textbf{A}
\begin{small}
\begin{tabular}{|c|c|c|c|}
\hline
C & Emin & C & Emin \\
\hline
Amin & C & D7 & D7 \\
\hline
\end{tabular}
\end{small}

\textbf{B}
\begin{small}
\begin{tabular}{|c|c|c|c|}
\hline
C & E7 & F & Fmin \\
\hline
C & F & Fmin & F \\
\hline
Fmaj7 & Emin7 & Fmaj7 & Emin7 \\
\hline
Bb & Amin & G & F \\
\hline
\end{tabular}
\end{small}

\end{columns}

\end{frame}

\end{document}
