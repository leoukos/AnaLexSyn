\documentclass{beamer}

\usepackage[utf8]{inputenc}
\usepackage[frenchb]{babel}
\usepackage[T1]{fontenc}

\usetheme{Singapore}

\begin{document}
\title{Analyse Lexicale et Syntaxique\\
	   } 
\author{Ken Déguernel - François Deslandes} 
\date{\today} 

\frame{\titlepage} 

\frame{\frametitle{Table of contents}\tableofcontents} 


\section{Présentation} 
\begin{frame}
\frametitle{Objectifs}
L'objectif de ce projet est d'appliquer les outils étudiés lors du cours d'Analyse Lexicale et Syntaxique à un exemple concret.\\
Le projet se décompose en deux parties :
\begin{enumerate}
\item Analyse Lexicale : comportant une partie de nettoyage et une partie de reconnaissance d'un texte. \\
\item Analyse Syntaxique : consistant en une vérification de l'intégrité et de la cohérence du texte. \\
\end{enumerate}
\end{frame}

\begin{frame}
\frametitle{Sujet}
Nous souhaitons produire de manière automatique des fichiers de partitions de jazz en format pdf, à partir d'un fichier source en texte brut.\\
Le fichier source comprend un certain nombre d'informations sur le morceau :
\begin{itemize}
\item Nom du morceau, auteur, style,
\item Mesuration, tonalité,
\item Thèmes (liste d'accords).
\end{itemize}
\end{frame}

\begin{frame}[fragile]
\frametitle{Exemple}
\begin{columns}

\column{1.5in}
\begin{tiny}
\begin{verbatim}
TITLE "Space Oddity" .Accompagnement guitare 
COMPOSER "David Bowie"
STYLE "Rock"
MESURATION 4/4
TONE "C"
PROG "AAB"

A/
C|Emin|C|Emin|
Amin|C|D7|D7|
\A


B/
C|E7|F|Fmin|
C|F|Fmin|F|
Fmaj7|Emin7|Fmaj7|Emin7|
Bb|Amin|G|F|
\B
\end{verbatim}
\end{tiny}

\column{1.5in}
\begin{center}
\textbf{Space Oddity}
\end{center}
\begin{small}
\begin{center}
\textbf{David Bowie}
\end{center}


Style : Rock

Mesuration : 4/4

Tone : C

Progression : AAB
\end{small}

\textbf{A}
\begin{small}
\begin{tabular}{|c|c|c|c|}
\hline
C & Emin & C & Emin \\
\hline
Amin & C & D7 & D7 \\
\hline
\end{tabular}
\end{small}

\textbf{B}
\begin{small}
\begin{tabular}{|c|c|c|c|}
\hline
C & E7 & F & Fmin \\
\hline
C & F & Fmin & F \\
\hline
Fmaj7 & Emin7 & Fmaj7 & Emin7 \\
\hline
Bb & Amin & G & F \\
\hline
\end{tabular}
\end{small}

\end{columns}
\end{frame}



\section{Analyse lexicale}
\begin{frame}
\frametitle{Analyse lexicale}
L'analyse lexicale consiste à :
\begin{enumerate}
\item Nettoyer le texte (commentaires, espacements surnuméraires, lignes vides,...),\\
\item Repérer les informations du morceau (Titre, Compositeur,...),\\
\item Reconnaitre les thèmes et les accords,\\
\item Pouvoir reconnaitre du texte.
\end{enumerate}
\end{frame}

\begin{frame}[fragile]
\frametitle{Reconnaitre les informations}
\begin{tiny}
\begin{verbatim}
TITLE         {return TITLE;}
COMPOSER      {return COMPOSER;}
MESURATION    {return MESURATION;}
TONE          {return TONE;}
PROG          {return PROG;}
STYLE         {return STYLE;}
\end{verbatim}
\end{tiny}
\end{frame}

\begin{frame}[fragile]
\frametitle{Reconnaitre les informations}
Un accord est composé d'une basse (une note de A à G) et d'une nature (par exemple : mineure, majeure,...)
\footnote{\tiny{La notation A-G est la notation américaine pour les notes do,ré,mi,...}\\}
\begin{tiny}
\begin{verbatim}
NATURE      (Maj|min|sus4|alt|aug|6|7|9|11|13|b|#|o|-|\+)
CHORD       ([A-G]{NATURE}*|%)
MES         [0-9][/][0-9]
TEXT        \"([a-z]|[ ])*\"
PROG        \"[A-B]*\"
TONEVAL     \"[A-G]\"
\end{verbatim}
\end{tiny}
\end{frame}




\section{Analyse syntaxique}
\begin{frame}

\end{frame}

\section{Résultats}
\begin{frame}

\end{frame}

\end{document}
